
























\begin{table}[htbp]
\caption{Minimum Requirements for Automatic Readmission into the Commerce Faculty}
\centering
\begin{tabular}{@{}p{0.12\textwidth}*{4}{L{\dimexpr0.22\textwidth-2\tabcolsep\relax}}@{}}
\toprule
& \multicolumn{2}{c}{BCom} &
\multicolumn{2}{c}{B.Bus.Sci} \\
\cmidrule(r{4pt}){2-3} \cmidrule(l){4-5}
& Number of courses required to pass & Cumulative Total of Courses & Number of courses &         Cumulative Total of Courses\\
\midrule
First year & 4 & 8 & 4 & 18 \\
\bottomrule
\end{tabular}
\label{table:mr}
\end{table}

\begin{table}[h!]
\caption{Minimum Requirements for Automatic Readmission
  into the Commerce Faculty} \label{table:mr}
\begin{tabularx}{\textwidth}{|l|Y|Y|Y|Y|}
\hline
&\multicolumn{2}{c|}{\bfseries B. Com}
&\multicolumn{2}{c|}{\bfseries B. Bus. Sci} \\ \hline
& Number of courses required to pass
& Cumulative Total of Courses
& Number of courses required to pass
& Cumulative Total of Courses   \\ \hline
First  Year &  4 &  8 &  4 & 18 \\ \hline
Second Year & 10 & 16 & 11 & 16 \\ \hline
Third  Year & 18 & 24 & 20 & 25 \\ \hline
\end{tabularx}
\end{table}

\begin{table}[htbp]
\caption{Minimum Requirements for Automatic Readmission into the Commerce Faculty}
\centering
\begin{tabular}{p{0.2\textwidth}p{0.2\textwidth}p{0.2\textwidth}p{0.2\textwidth}p{0.2\textwidth}} \\ \toprule

& \multicolumn{2}{c}{BCom} & \multicolumn{2}{c}{B. Bus Scii} \\ \midrule
& Number of courses required to pass & Cumulative Total of Courses & Number of courses & Cumulative Total of Courses \\ \midrule
First year & 4 & 8 & 4 & 8 \\
Second year & 10 & 16 & 11 & 16 \\
Third year & 18 & 24 & 20 & 25 \\ \bottomrule
\end{tabular}
\end{table}

\begin{table}[htbp]
\caption{Minimum Requirements for Automatic Readmission into the Commerce Faculty}
\centering
\begin{tabular}{p{0.18\textwidth}p{0.18\textwidth}p{0.18\textwidth}p{0.18\textwidth}p{0.18\textwidth}} \\ \toprule

& \multicolumn{2}{c}{BCom} & \multicolumn{2}{c}{B. Bus Scii} \\ \midrule
& \multicolumn{1}{p{3.5cm}}{Number of courses required to pass} & Cumulative Total of Courses & \multicolumn{1}{p{2cm}}{Number of courses} & Cumulative Total of Courses \\ \midrule
First year & 4 & 8 & 4 & 8 \\
Second year & 10 & 16 & 11 & 16 \\
Third year & 18 & 24 & 20 & 25 \\ \bottomrule
\end{tabular}
\end{table}

\begin{table}[ht]
\centering
\begin{tabular}{@{\extracolsep{4pt}}llccccccc}
\toprule
{} & {} & {Observations} & \multicolumn{3}{c}{Median}  & \multicolumn{3}{c}{SD}\\
 \cmidrule{3-3}
 \cmidrule{4-6}
 \cmidrule{7-9}
 Year & Group & N & V1 & V2 & V3 & V1 & V2 & V3 \\
\midrule
2012  & Control & 2 & 0.052 & 0.294 & 0.115 & 0.304 & 0.619 & 0.611 \\
  & Treat & 2 & 0.511 & 0.083 & 0.123 & 0.573 & 0.541 & 0.734 \\
2016  & Control & 3 & 0.320 & 0.344 & 0.382 & 0.382 & 0.494 & 0.477 \\
  & Treat & 3 & 0.378 & 0.296 & 0.123 & 0.386 & 0.668 & 0.732 \\
\bottomrule
\end{tabular}
\caption{Mean by Year and Group}
\end{table}























\begin{enumerate}
\item Not valid
\item Not valid
\item Valid, 1 row 2 columns or 1x2
\item Valid, 2 rows 1 column or 2x1
\item Not Valid
\item Valid, 1 row and 1 column or 1x1
\item Valid, 3 rows and 3 columns or 3x3
\end{enumerate}







The ray equation or the equation of a point along a line is defined as $P=O+tR$ For two points we can change it as such:
$P=v_0 + t(v_1 - v_0)$ where $v_0$ and $v_1$ are two points.

inserting our points
$[-3/2,2]$
$[3,0]$

and thus the equation becomes

$P=v_0 + t(v_1 - v_0) = \begin{bmatrix} -3/2 \\ 2\end{bmatrix} + t\left(\begin{bmatrix} 3 \\ 0\end{bmatrix} - \begin{bmatrix} -3/2 \\ 2\end{bmatrix}\right) = \begin{bmatrix} -3/2 \\ 2\end{bmatrix} + t\begin{bmatrix} 9/2 \\ -2\end{bmatrix}$

lastly we'll try to find the cartesian equation.  Given two points $(x_1, x_2)$ and $(y_1, y_2)$ we can compute the slope of $a$ in the following way: $a = \frac{y_2 - y_1}{x_1 - x_2}$
thus:

\begin{equation}
a = \frac{0 - 2}{ 3 - (-\frac{3}{2})} = \frac{2}{\frac{9}{2}} = - \frac{4}{9}
\end{equation}

we substitute the slope with a random point,
The point i've chosen is:
\[
(x_1, y_1) = \left ( -\frac{3}{2}, 2 \right )
\]

back to substitution:

\begin{equation}
y -2 = -\frac{4}{9}\left ( x - \left ( -\frac{3}{2} \right ) \right )
\end{equation}

we then solve for y:
\begin{equation}
y = -\frac{4}{9}(x-3)
\end{equation}

our equation is then:
\begin{equation}
y = \frac{4}{3} - \frac{4x}{9}
\end{equation}

a bit of cleaning up and we can distil the equation to it's purest form:

\begin{equation}
  \frac{4x}{9} + y = \frac{4}{3}
\end{equation}


\begin{equation}
9\left ( \frac{4x}{9} + y \right ) = \frac{9 \cdot 4}{3}
\end{equation}

\begin{equation}
4x+9y=\frac{9\cdot 4}{3}
\end{equation}

\begin{equation}
\frac{9 \cdot 4}{3} = 12
\end{equation}

\begin{equation}
4x + 9y = 12
\end{equation}













\begin{comment}

Template for source code inclusion:

\begin{lstlisting}
import numpy as np

CPRN = 63952537
np.random.seed(CPRN)
np.set_printoptions(precision=3)

def generate_data():
    a = np.array([[1, 3, 1], [2, 1, 1], [-2, 2, -1]])
    x = np.linalg.det(a)
    y = np.linalg.inv(a)
    print(x)
    print(y)

d = np.random.randint(10000, 100000, 3)
generate_data()

print(d)
\end{lstlisting}
\end{comment}


\begin{comment} %------------- Matrix Templates and math -------------------------
$
\begin{bmatrix}
  a & b \\ c & d
\end{bmatrix}
$

\[
\begin{bmatrix}
a & b \\ c & d
\end{bmatrix}
\]



\[    % <-- start math environment
R_z (\theta)=
\begin{bmatrix}
    \cos(\theta) & -\sin(\theta)  & 0 \\
    \sin(\theta) &  \cos(\theta)  & 0 \\
    0            & 0             & 1
\end{bmatrix}
\]    % <-- end of math environment

\end{comment}

\[
\begin{bmatrix}
    x_{11}       & x_{12} & x_{13} & \dots & x_{1n} \\
    x_{21}       & x_{22} & x_{23} & \dots & x_{2n} \\
    \hdotsfor{5} \\
    x_{d1}       & x_{d2} & x_{d3} & \dots & x_{dn}
\end{bmatrix}
=
\begin{bmatrix}
    x_{11} & x_{12} & x_{13} & \dots  & x_{1n} \\
    x_{21} & x_{22} & x_{23} & \dots  & x_{2n} \\
    \vdots & \vdots & \vdots & \ddots & \vdots \\
    x_{d1} & x_{d2} & x_{d3} & \dots  & x_{dn}
\end{bmatrix}
\]

{\renewcommand\labelitemi{}
\begin{itemize}
\item Step (1)
\end{itemize}


Template for figure inclusion:

%\begin{figure}[htb]
%\begin{center}
%% uncomment the line below and change the figure file.
%  \includegraphics[width=1.4\textwidth]{ex1}
%\end{center}
%\end{figure}

%%%%%%%%%%%%%%%%%%%%%%%%%%%%%%%%%%%%%%%%%%%%%%%%%%%%%%%%%%%%
%%%%%%%%%%%%%%%%%%%%%%%%%%%%%%%%%%%%%%%%%%%%%%%%%%%%%%%%%%%%
\newpage
\section*{Exercise 2}
%%%%%%%%%%%%%%%%%%%%%%%%%%%%%%%%%%%%%%%%%%%%%%%%%%%%%%%%%%%%
%%%%%%%%%%%%%%%%%%%%%%%%%%%%%%%%%%%%%%%%%%%%%%%%%%%%%%%%%%%%
