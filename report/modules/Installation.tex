\section{Installing MongoDB}\label{installing-mongodb}

Below you'll find the installation instructions on the following platforms.

\begin{itemize}
\item
 ~\nameref{sec:wsl}
\item
 ~\nameref{sec:16.04}
\item
  ~\nameref{sec:14.04}
\end{itemize}

\leavevmode
\newline
\subsection{WSL(Windows Subsystem for Linux)}\label{sec:wsl}

Because services are not yet supported in WSL, we must instead install mongodb version 2.6.10 \texttt{mongodb}. All we have to do is call:
\begin{lstlisting}[backgroundcolor = \color{light-gray}]
$ sudo apt-get install mongodb
\end{lstlisting}
~\\
Alternatively an included bash script that contains exactly the above is also included:
\begin{lstlisting}[backgroundcolor = \color{light-gray}]
$ sudo bash install-Mongodb-on-WLS.sh
\end{lstlisting}
~\\
The drawback here is that we are not using the official package provided by the MongoDB organization: \texttt{mongodb-org}.

\leavevmode
\newline
\subsection{Ubuntu 16.04, 17.04(not tested), 18.04}\label{sec:16.04}

The following was taken from here:
\\
\url{https://docs.mongodb.com/manual/tutorial/install-mongodb-on-ubuntu/}
\\
\\
First we need to add the RSA public key to our package manager:
\begin{lstlisting}[backgroundcolor = \color{light-gray}]
$ sudo apt-key adv --keyserver hkp://keyserver.ubuntu.com:80 --recv 2930ADAE8CAF5059EE73BB4B58712A2291FA4AD5
\end{lstlisting}
~\\
Then we need to add to our package manager; the location of where to find the MongoDB binaries.
\begin{lstlisting}[backgroundcolor = \color{light-gray}]
$ echo "deb [ arch=amd64,arm64 ] https://repo.mongodb.org/apt/ubuntu xenial/mongodb-org/3.6 multiverse" | sudo tee /etc/apt/sources.list.d/mongodb-org-3.6.list
\end{lstlisting}
~\\
We then need to reload our package manager.
\begin{lstlisting}[backgroundcolor = \color{light-gray}]
$ sudo apt-get update
\end{lstlisting}
~\\
Then we install \texttt{mongodb-org}.
\begin{lstlisting}[backgroundcolor = \color{light-gray}]
$ sudo apt-get install -y mongodb-org
\end{lstlisting}
~\\
Alternatively the following shell script contains all the above:
\begin{lstlisting}[backgroundcolor = \color{light-gray}]
$ sudo bash install-Mongodb-on-16.04-and-above.sh
\end{lstlisting}
~\\


\leavevmode
\newline
\subsection{Ubuntu 14.04}\label{sec:14.04}
The following was taken from here:
\\
\url{https://docs.mongodb.com/manual/tutorial/install-mongodb-on-ubuntu/}
\\
\\
First we need to add the RSA public key to our package manager:
\begin{lstlisting}[backgroundcolor = \color{light-gray}]
$ sudo apt-key adv --keyserver hkp://keyserver.ubuntu.com:80 --recv 2930ADAE8CAF5059EE73BB4B58712A2291FA4AD5
\end{lstlisting}
~\\
Then we need to add to our package manager; the location of where to find the MongoDB binaries.
\begin{lstlisting}[backgroundcolor = \color{light-gray}]
$ echo "deb [ arch=amd64 ] https://repo.mongodb.org/apt/ubuntu trusty/mongodb-org/3.6 multiverse" | sudo tee /etc/apt/sources.list.d/mongodb-org-3.6.list
\end{lstlisting}
~\\
We then need to reload our package manager.
\begin{lstlisting}[backgroundcolor = \color{light-gray}]
$ sudo apt-get update
\end{lstlisting}
~\\
Then we install \texttt{mongodb-org}.
\begin{lstlisting}[backgroundcolor = \color{light-gray}]
$ sudo apt-get install -y mongodb-org
\end{lstlisting}
~\\
Alternatively the following shell script contains all the above:
\begin{lstlisting}[backgroundcolor = \color{light-gray}]
$ sudo bash install-Mongodb-on-14.04.sh
\end{lstlisting}
